
\authoredSection{michael}{Fazit}

Das Projekt objektorientierte Softwareentwicklung hat für uns in gewisser Hinsicht eine Zäsur im Studium markiert. Bisherige Module oder Praktika legten den Fokus auf eine technisch möglichst saubere Implementierung, bei der im Zweifel das Interaktionsdesign oder der Funktionsumfang zurückgestellt wurden. Im Projekt wurde die Beherrschung objektorientierter Techniken bereits vorausgesetzt und erstmalig ein Software-Produkt entwickelt, das in seiner Gesamtheit auch industriellen Anforderungen genügen musste. 

Mehrere Teilnehmer unserer Gruppe hatten bereits ein erstes Maß an Arbeitserfahrung vorzuweisen, etwa als Werkstudent oder freiberuflicher Webentwickler. Trotzdem wurde in den ersten Wochen des Projekts deutlich, dass im Bezug auf die Arbeitsorganisation als Auftragnehmer eines Produktes noch Raum für Verbesserungen war. 

Dies galt insbesondere für den Umgang mit Designfragen und der Reihenfolge der Entwicklung. In einer Gruppe, die bis auf eine Ausnahme aus Informatikstudenten bestand, gingen wir zunächst in aus anderen Modulen vertrauter Manier vor. Um die Eigenheiten von iOS und Objective C kennenzulernen, unternahmen wir erste Gehversuche in Funktionen wie einer Überweisungsansicht, die für das Endprodukt sekundär waren. Ebenso behandelten wir in den ersten Überlegungen das Design noch etwas stiefmütterlich. Eine Mentalität, der im Informatikstudium nicht selten Vorschub geleistet wird, ließe sich auf die Aussage `"Erst wird programmiert, dann das Design nachgeschoben"' reduzieren. Im Plenum wurde uns aufgezeigt, dass ein solcher Ansatz nicht nur weniger effizient ist, sondern auch das eigene Produkt schlechter dastehen lässt. 

Denn einerseits ist es in einem engen Zeitrahmen, der ohnehin eher auf die Entwicklung eines Prototyps abzielt, sinnvoller, sich voll und ganz auf die Funktionen zu konzentrieren, die Alleinstellungsmerkmale darstellen und den innovativen Kern des Konzepts bilden. Denn nur anhand dieser lässt sich ein Kunde von einem Produkt überzeugen. Zudem erfordert ein benutzerzentriertes Design, das erst nachträglich hinzugefügt wird, umständliche Änderungen an der Codebasis. Gleichzeitig können bereits wenige, minimalistische Designelemente in konsistenter Farbe und Anordnung einen enormen Unterschied in der Wahrnehmung der Produktqualität bedeuten.

Entsprechend haben wir unsere Arbeitsweise adjustiert und ein neues Verständnis von zeitgemäßer Softwareentwicklung gewonnen. Während innovative Funktionalität von technischer Seite weiter geboten ist, hat sich der Anspruch der Benutzer an die Benutzbarkeit stark gewandelt; Apps, die dies nicht berücksichtigen, haben auf einem derart umkämpften Markt wenig Erfolgsaussicht.

