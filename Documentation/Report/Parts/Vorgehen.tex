\authoredSection{alex}{Vorgehen}
\subsection{Scrum}
	Im Rahmen des Projekts wurden wir mit der Srum-Methodik vertraut gemacht und hatten Gelegenheit,eine einführende Zertifizierung zu absolvieren. Während der erste Hälfte des Semesters standen Vision, Konzeption und Lernprozesse im Vordergrund. Mit Beginn der eigentlichen Implementierung in der zweiten Semesterhälfte bestand dann die Herausforderung darin, den Scrum-Prozess auf die zeitlichen Gegegebenheiten einer Gruppe von Studenten mit unterschiedlichen Stundenplänen abzustimmen. Zunächst einmal war es, abgesehen von Ausnahmen, kaum möglich, sich außerhalb der festen Projekttermine in voller Gruppenstärke zu einem festen Termin zu treffen. 
    
	Um in unseren Arbeitsabläufen dennoch eine gewisse Kontinuität herzustellen, haben wir zwei wöchentliche Termine festgelegt, bei denen immer mindestens die Hälfte der Gruppe anwesend sein konnte. Als Sprintdauer haben wir zwei Wochen festgelegt, hauptsächlich basierend auf der Erfahrung, wie lange einzelne Features in anderen universitären Projekten gedauert haben. In gewisser Hinsicht haben wir hier also eine Scrum-ähnliche Retrospektive benutzt, um unseren ersten Sprint zu planen.
    
	Das Scrum-Framework verbietet eine Aufteilung in Unterteams. Zugleich wurde uns aber vermittelt, die Wegnahme einzelner Scrum-Elemente oder Missachtung einzelner Regeln bedeute, gar nicht mehr Scrum zu benutzen, da Scrum unteilbar sei. Ein weiteres Problem ergab sich dadurch, dass Scrum unserem Verständnis nach vorsieht, dass das Entwicklungsteam zu Beginn der Arbeit bereits alle notwendigen technischen Kompetenzen zur Umsetzung eines Projekts besitzt. Zuverlässige Schätzungen für die Entwicklungszeit sind andernfalls schwer möglich. Universitäre Projekte haben aber natürlich auch immer eine Komponente, in der sich die Teilnehmer selbstständig das Wissen erarbeiten, das zur Vollendung einer Aufgabe notwendig ist. Diese Überlegung haben wir in unsere Schätzung mit einbezogen. Dennoch ist es an technisch schwierigen Stellen schwer abzusehen, wie lange es dauern wird, eine spezielle Lösung zu finden. Insofern sind Hilfen wie Burdown-Charts dann nicht besonders indikativ dafür, wie sich bisher Erledigtes zu verbleibenden Aufgaben mit Blick auf die Einarbeitungsschwierigkeiten verhält.